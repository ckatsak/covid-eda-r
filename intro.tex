\section{Introduction}

This work is a basic Exploratory Data Analysis of the COVID-19 Data provided by the Center for Systems Science and Engineering (CSSE) at Johns Hopkins University\cite{ghcsse}.

For the development, I leveraged the power of the R language on a GNU/Linux \texttt{x86\_64} platform, as provided through the official \texttt{r-base} Docker images\cite{dhrbase} of the project (which are based on the Debian \texttt{testing} release), as well as the \texttt{Vim}\cite{vim} text editor.

The development has relied upon the modern principles of structured programming and taken place in a modular, function-based fashion, rather than employing a non-modular, top-down scripting process.
As a disclaimer, and with respect to this assignment's requirements to include the code within this document \textit{and} to not exceed the limit of $15$ pages, this practice might sometimes be convenient, but it might as well sometimes be not.
%TODO: repo
In any case, the source code has been made publicly available in its whole for further inspection; it is hosted on GitHub and can be accessed at \url{https://github.com/ckatsak/covid-eda-r}.
%TODO: repo